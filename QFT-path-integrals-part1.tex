\documentclass[11pt]{article}
\usepackage[margin=0cm,nohead]{geometry}
\usepackage[active,tightpage]{preview}

\usepackage{tikz,amsmath,amssymb,bm,color,cancel}

%\usepackage{newtxtext}
%\usepackage{newtxmath}

\usetikzlibrary{shapes,arrows}
\usetikzlibrary{calc}
\usetikzlibrary{positioning}
\usetikzlibrary{decorations.pathreplacing}
\usetikzlibrary{decorations.markings}

\PreviewEnvironment{tikzpicture}
\setlength\PreviewBorder{-1mm}

\begin{document}

\begin{tikzpicture}

%%%%%%%%%%%%%%%%%%%%%%%%%%%%%%%%%%%%%%%%%%%%%%%%%%%%%%%%%%%%%% Macros

\newcommand{\bb}[2]{
	\!\!{\tt #1}
	\\[6pt]
	~{\large \color{blue} #2}
}
\newcommand{\bbb}[3]{
	\!\!{\tt #1}
	\\[6pt]
	~{\large \color{blue} #2}
	\\[6pt]
	~{#3}
}

\newcommand{\tr}{\mathsf{{\scriptscriptstyle T}}}
\newcommand{\ii}{\mathrm{i}}
\newcommand{\ee}{\operatorname{e}}
\newcommand{\dd}{\mathrm{d}}
\newcommand{\op}[1]{\operatorname{#1}}

\newcommand{\pL}{\mathsf{{\scriptscriptstyle L}}}
\newcommand{\pR}{\mathsf{{\scriptscriptstyle R}}}
\newcommand{\pC}{\mathsf{{\scriptscriptstyle C}}}

\newcommand{\bPsi}{\boldsymbol{\Psi}}
\newcommand{\bPhi}{\boldsymbol{\Phi}}

\newcommand{\suY}{\mathrm{\scriptscriptstyle Y}}
\newcommand{\suL}{\mathrm{\scriptscriptstyle L}}
\newcommand{\suC}{\mathrm{\scriptscriptstyle C}}

\newcommand{\dbar}[1]{\overline{#1}}
\newcommand{\sbar}[1]{\bar{#1}}
\newcommand{\anti}[1]{\bar{#1}}

\newcommand{\quantNo}[3]{\mathbf{#1},\mathbf{#2},#3}
 
%%%%%%%%%%%%%%%%%%%%%%%%%%%%%%%%%%%%%%%%%%%%%%%%%%%%%%%%%%%%%% Varying Box Dim

% Define VaryBox style of a size between two nodes

\newcommand*{\SetVaryBox}[2]{
	\path (#1); \pgfgetlastxy{\endAx}{\endAy};
	\path (#2); \pgfgetlastxy{\endBx}{\endBy};
	\tikzset{VaryBox/.style={
		minimum height = \endAy - \endBy,
		minimum width  = \endBx - \endAx,
		text width     = \endBx - \endAx - 3mm,
		inner sep      = 0
	}};
}

%%%%%%%%%%%%%%%%%%%%%%%%%%%%%%%%%%%%%%%%%%%%%%%%%%%%%%%%%%%%%% Box styles

\tikzset{myRow/.style     ={draw,anchor=north west,rectangle}}
\tikzset{boxWhite/.style  ={draw,anchor=north west,rectangle, very thin, rounded corners=5pt, fill=white}}
\tikzset{boxGray/.style   ={draw,anchor=north west,rectangle, very thin, rounded corners=5pt, fill=gray!10}}
\tikzset{boxRed/.style    ={draw,anchor=north west,rectangle, very thin, rounded corners=5pt, fill=red!4}}
\tikzset{boxGreen/.style  ={draw,anchor=north west,rectangle, very thin, rounded corners=5pt, fill=green!2}}
\tikzset{boxYellow/.style ={draw,anchor=north west,rectangle, very thin, rounded corners=5pt, fill=yellow!5}}
\tikzset{boxBlue/.style   ={draw,anchor=north west,rectangle, very thin, rounded corners=5pt, fill=blue!2}}
\tikzset{boxCyan/.style   ={draw,anchor=north west,rectangle, very thin, rounded corners=5pt, fill=cyan!2}}
\tikzset{colorLagr/.style ={fill=magenta!8}}

%%%%%%%%%%%%%%%%%%%%%%%%%%%%%%%%%%%%%%%%%%%%%%%%%%%%%%%%%%%%%% Row/Column styles

\tikzset{R/.style={
	minimum height = #1 mm
}}

\tikzset{C/.style={
    minimum width = #1 mm, 
    text width = #1 mm - 3 mm,
    inner sep = 0
}}

%%%%%%%%%%%%%%%%%%%%%%%%%%%%%%%%%%%%%%%%%%%%%%%%%%%%%%%%%%%%%%%%%%%%%%%%%%%%%%%

\tikzset{cross/.style={
	cross out, thick, draw=black, inner sep=0pt, outer sep=0pt,
	minimum size=1.5mm}
};

%%%%%%%%%%%%%%%%%%%%%%%%%%%%%%%%%%%%%%%%%%%%%%%%%%%%%%%%%%%%%% Paper W/H

%\draw [blue!40] (-20.5,14.5) rectangle (20.5,-14.5);

\node at (-20.5,14.5) {}; \node at (20.5,-14.5) {};

%%%%%%%%%%%%%%%%%%%%%%%%%%%%%%%%%%%%%%%%%%%%%%%%%%%%%%%%%%%%%% Title

\node [color=black] at (0,14.25) {
	\huge\sf Path Integrals: Perturbation expansion for the $\phi^4$-theory
};

\node [anchor=north, color=black] at (0,13.9) {
	by M.B.Kocic ~--~ Version 1.06 (2016-03-28) ~--~ FK8017 HT15
};

\pdfinfo {
%	/CreationDate	(D:20160121124500)
	/Title			(Path Integrals: Perturbation expansion for the phi^4 theory)
	/Author			(Mikica B Kocic)
	/Subject		(FK8017 HT15)
	/Keywords		(QFT SM)
	/Creator		(TeX-TikZ)
}

%%%%%%%%%%%%%%%%%%%%%%%%%%%%%%%%%%%%%%%%%%%%%%%%%%%%%%%%%%%%%% Minkowski

\node [boxGreen,anchor=north west,R=25,C=40,align=center] at (-20.5,13) {\Large
	\textsf{\textbf{Lorentzian\\[2mm]Path Integral}}
};

\node [boxWhite,anchor=north west,R=25,C=115,align=center] at (-16,13) {\vspace{-5mm}
	\begin{gather*}
	\textstyle \int\dd^4 x  \equiv \int\dd x^0 \dots \int\dd x^3, \qquad
	\int\dd^4 p  \equiv \int\dd p^0 \dots \int\dd p^3, \\[1mm]
	\square \equiv \partial_0^2 - \nabla^2
	= \partial_0^2 - \partial_1^2 - \partial_2^2 - \partial_3^2, \\[1mm]
	x^2 \equiv (x^0)^2 - \mathbf{x}^2, \quad
	p^2 \equiv (p^0)^2 - \mathbf{p}^2, \quad
	p\cdot x \equiv p^0 x^0 - \mathbf{p}\cdot\mathbf{x}
	\end{gather*}
};

\node [anchor=south west] at (-20.5,9.75) {
	\textsf{\textbf{ Action}}
};

\node [boxYellow,R=22.5,minimum width=160mm, text width=157mm,align=left] at (-20.5,9.75) {\vspace{-4mm}
	\begin{alignat*}{3}
	\mathcal{L}[\phi] &= \mathcal{L}_0[\phi] + \mathcal{L}_\mathrm{int}[\phi], 
	\qquad& \mathcal{L}_0[\phi] &= \tfrac{1}{2} (\partial_\mu \phi) (\partial^\mu \phi) - \tfrac{1}{2}m^2 \phi^2,
	\qquad& \mathcal{L}_\mathrm{int}[\phi] &= - \tfrac{1}{4!} \lambda \, \phi^4, \\[2mm]
	S[\phi] &= S_0[\phi] + S_\mathrm{int}[\phi], 
	& S_0[\phi] &= \int\!\dd^4 x \,\mathcal{L}_0[\phi],
	& S_\mathrm{int}[\phi] &= \int\!\dd^4 x \,\mathcal{L}_\mathrm{int}[\phi] \,.
	\end{alignat*}
};

\node [anchor=south west] at (-20.5,6.5) {
	\textsf{\textbf{ Feynman propagator}}
};

\node [boxYellow,R=35,C=85,align=center] at (-20.5,6.5) {\vspace{-5mm}
	\begin{align*}
	\Delta_{\mathrm{F}}(x_1-x_2) &= \int_{C_{\mathrm{F}}} \frac{\dd^4 p}{(2\pi)^4}
	   \ee^{-\ii p \cdot (x_1-x_2)}
	   \frac{1}{p^2-m^2} 
	\\[2mm]
	\ii \Delta_{\mathrm{F}}(x_1-x_2) &= \left\langle 0 \right\vert \mathrm{T} \{ \, \hat\phi(x_1), \, \hat\phi(x_2) \, \} \left\vert 0 \right\rangle 
	\\[2mm]
	\ii \Delta_{\mathrm{F}}(x_1-x_2) &= \Delta^\mathrm{E}_{\mathrm{F}}(x_1-x_2)
	\end{align*}
};

\node [anchor=south west] at (-20.5,1.5) {
	\textsf{\textbf{ Generating Functional}}
};

\node [boxYellow,R=25,C=93,align=center] at (-20.5,1.5) {\vspace{-4mm}
	\begin{align*}
	Z_0[J] &= \int\!\mathcal{D}\phi\, \exp\left( \ii\, S_0 + \ii\int\!\dd^4 z\, J(z)\phi(z)\right)
	\\[0.5mm]
	Z[J] &= \int\!\mathcal{D}\phi\, \exp\left( \boxed{\ii\, S_\mathrm{int}} + \ii\, S_0
	    + \ii\!\int\!\dd^4 z\, J(x)\phi(z) \right)
	\end{align*}
};

\node [boxYellow,R=32.5,C=108,align=center] at (-11,1.5) {\vspace{-4mm}
	\begin{alignat*}{3}
	Z_0[J] &= Z_0[0] \, \exp\left( -\frac{\ii}{2} \int\!\dd^4 z_1 \, \dd^4 z_2\, J(z_1) \Delta_\mathrm{F}(z_1\!-\!z_2) J(z_2) \right)
	\\[0.5mm]
	Z[J] &= \exp\left( \,\boxed{\ii \, S_\mathrm{int}} \left[\frac{\delta}{\ii\,\delta J(y)} \right] \right) Z_0[J]
	\\[1mm]
	\quad Z_0[0] &= \textstyle \int \mathcal{D}\phi \, \exp(\ii S_0) 
	= \left( \det (\square + m^2) \right)^{-1/2}
	\end{alignat*}
};

\node at (-11,0.25) {\large $\implies$};

\node [anchor=south west] at (-20.5,-2) {
	\textsf{\textbf{ $n$-point correlation function}}
};

\node [boxYellow,R=15,C=105,align=center] at (-20.5,-2) {\vspace{-4mm}
	\begin{align*}
	G^{(n)}(x_1,\dots,x_n) 
	=  \left( \frac{1}{Z[J]} \frac{\delta}{\ii\delta J(x_1)} \cdots \frac{\delta}{\ii\delta J(x_n)} Z[J] \right)_{J=0}
	\end{align*}
};

\node [rotate=-90] at (-10.65,-1.9) {\large $\implies$};

\node [boxWhite,R=15,C=110,anchor=north,align=center] at (0,-2) {\vspace{-4mm}
	\begin{gather*}
	G^{(n)}(x_1,\dots,x_n) = G_\mathrm{E}^{(n)}(x_1,\dots,x_n),
	\\
	G^{(2)}_0(x_1,x_2) 
	=  \ii \Delta_\mathrm{F}(x_1-x_2)
	= \Delta^\mathrm{E}_\mathrm{F}(x_1-x_2)
	= G_{0,\mathrm{E}}^{(2)}(x_1,x_2) 
	\end{gather*}
};

%%%%%%%%%%%%%%%%%%%%%%%%%%%%%%%%%%%%%%%%%%%%%%%%%%%%%%%%%%%%% Wick Rotation

\node [boxRed,anchor=north west,R=10,C=80,align=center] at (-4,13) {\large
	\textsf{\textbf{The Wick Rotation}} ~~ $\ii S_\mathrm{L} = - S_\mathrm{E}$
};

\node [boxWhite,anchor=north west,R=45,C=80,align=center] at (-4,12) {
	\vspace{-5mm}
	\begin{gather*}\textstyle
	x^0 \equiv -\ii x^4, \ x^4 \equiv \ii x^0, \quad
	p^0 \equiv \ii p^4, \ p^4 \equiv -\ii p^0, \\[1mm]
	\partial_0 = \ii \partial_4, \quad
	\dd x^0 = -\ii \dd x^4, \quad
	\square_\mathrm{L} = -\square_\mathrm{E},
	\\[1mm]
	\textstyle \int\dd^4 x_\mathrm{L} = -\ii \int\dd^4 x_\mathrm{E}, \quad
	\textstyle \int\dd^4 p_\mathrm{L} = \ii \int\dd^4 p_\mathrm{E},
	\\[1mm]
	x^2_\mathrm{L} = -x^2_\mathrm{E}, \quad
	p^2_\mathrm{L} = -p^2_\mathrm{E},
	\\[1mm]
	p^0 x^0 = p^4 x^4, \quad
	(p\cdot x)_\mathrm{L} \neq (p\cdot x)_\mathrm{E},\\[1mm]
	\delta(x^0) \text{ using FT def.} = \ii \delta(x^4) \,.
	\end{gather*}
};

\node [anchor=north west,R=30,C=90,align=justify] at (-4.5,7.25) {
	Note: The signs are chosen such that the product of 
	time and energy, $x^0p^0=x^4p^4$, is invariant.
	This is an arbitrary choice which ensures that the travelling 
	direction of a plane wave remains unchanged when the coordinates 
	are transformed.
	\\[2mm]\color{black!70!green}
	For an arbitrary $f(p^2)$, it holds:\vspace{-2mm}
	\begin{align*}
	\int\!\frac{\dd^4 p}{(2\pi)^4}
	   \ee^{-\ii p \cdot x}f(p^2)
	= \ii\!\int\!\frac{\dd^4 p_\mathrm{E}}{(2\pi)^4}
	   \ee^{-\ii p_\mathrm{E} \cdot x_\mathrm{E}} f(-p^2_\mathrm{E}) 
	\end{align*}
	Since $p\cdot x \neq p_\mathrm{E} \cdot x_\mathrm{E}$,
	the above relation requires the insensitivity under spatial reflections: 
	$\mathbf{p} \to -\mathbf{p}$.
};

\node [anchor=north west,R=5,C=125,align=left] at (7.75,2.85) {
	\textsf{\textbf{\color{red}
	The euclidean causal propagator does not require 
	the $\ii \varepsilon$-prescription!}}
};

%%%%%%%%%%%%%%%%%%%%%%%%%%%%%%%%%%%%%%%%%%%%%%%%%%%%%%%%%%%%% Euclidean

\node [boxGreen,anchor=north west,R=25,C=40,align=center] at (16.5,13) {\Large
	\textsf{\textbf{Euclidean\\[2mm]Path Integral}}
};

\node [boxWhite,anchor=north west,R=25,C=115,align=center] at (4.5,13) {\vspace{-5mm}
	\begin{gather*}
	\textstyle \int\dd^4 x  \equiv \int\dd x^1 \dots \int\dd x^4, \qquad
	\int\dd^4 p  \equiv \int\dd p^1 \dots \int\dd p^4, \\[1mm]
	\square \equiv \partial_4^2 + \nabla^2
	= \partial_1^2 + \partial_2^2 + \partial_3^2 + \partial_4^2, \\[1mm]
	x^2 \equiv \mathbf{x}^2 + (x^4)^2, \quad
	p^2 \equiv \mathbf{p}^2 + (p^4)^2, \quad
	p\cdot x \equiv \mathbf{p}\cdot\mathbf{x} + p^4 x^4
	\end{gather*}
};

\node [anchor=south east] at (20.5,9.75) {
	\textsf{\textbf{ Action}}
};

\node [boxBlue,R=22.5,minimum width=160mm, text width=157mm,align=left] at (4.5,9.75) {\vspace{-4mm}
	\begin{alignat*}{3}
	\mathcal{L}[\phi] &= \mathcal{L}_0[\phi] + \mathcal{L}_\mathrm{int}[\phi], 
	\qquad& \mathcal{L}_0[\phi] &= \tfrac{1}{2} (\partial_\mu\phi)(\partial^\mu\phi) 
	+ \tfrac{1}{2} m^2\phi^2,
	\qquad& \mathcal{L}_\mathrm{int}[\phi] &= \tfrac{1}{4!} \lambda \, \phi^4, \\[2mm]
	S[\phi] &= S_0[\phi] + S_\mathrm{int}[\phi], 
	& S_0[\phi] &= \int\!\dd^4 x \,\mathcal{L}_0[\phi],
	& S_\mathrm{int}[\phi] &= \int\!\dd^4 x \,\mathcal{L}_\mathrm{int}[\phi] \,.
	\end{alignat*}
};

\node [anchor=south east] at (20.5,6.5) {
	\textsf{\textbf{ Feynman propagator}}
};

\node [boxBlue,R=35,C=85,align=center] at (12,6.5) {\vspace{-5mm}
	\begin{align*}
	\Delta^{\mathrm{E}}_{\mathrm{F}}(x_1-x_2) &= \int 
	   \frac{\dd^4 p}{(2\pi)^4}
	   \ee^{-\ii p \cdot (x_1-x_2)}
	   \frac{1}{p^2+m^2}
	\\[2mm]
	\Delta^{\mathrm{E}}_{\mathrm{F}}(x_1-x_2) &= \left\langle 0 \right\vert \mathrm{T} \{ \, \hat\phi(x_1), \, \hat\phi(x_2) \, \} \left\vert 0 \right\rangle 
	\\[2mm]
	\Delta^\mathrm{E}_{\mathrm{F}}(x_1-x_2) &= \ii \Delta_{\mathrm{F}}(x_1-x_2)
	\end{align*}
};

\node [anchor=south east] at (20.5,1.5) {
	\textsf{\textbf{ Generating Functional}}
};

\node [boxBlue,R=25,C=95,align=center] at (11,1.5) {\vspace{-4mm}
	\begin{align*}
	Z_0[J] &= \int\!\mathcal{D}\phi\, \exp\left( - S_0 + \int\!\dd^4 z\, J(z)\phi(z)\right)
	\\[0.5mm]
	Z[J] &= \int\!\mathcal{D}\phi\, \exp\left( \boxed{- S_\mathrm{int}} - S_0  
	    + \int\!\dd^4 z\, J(x)\phi(z) \right)
	\end{align*}
};

\node [boxBlue,R=32.5,C=108,align=center] at (0,1.5) {\vspace{-4mm}
	\begin{alignat*}{3}
	Z_0[J] &= Z_0[0] \, \exp\left( \,\frac{1}{2}\int\!\dd^4 z_1 \, \dd^4 z_2\, J(z_1) \Delta^\mathrm{E}_\mathrm{F}(z_1\!-\!z_2) J(z_2) \right)
	\\[0.5mm]
	Z[J] &= \exp\left( \,\boxed{- S_\mathrm{int}}\left[\frac{\delta}{\delta J(y)} \right] \right) Z_0[J]
	\\[1mm]
	\quad Z_0[0] &= \textstyle \int \mathcal{D}\phi \, \exp(- S_0) 
		= \left( \det (-\square + m^2) \right)^{-1/2}
	\end{alignat*}
};

\node at (10.8,0.25) {\large $\Longleftarrow$};

\node [anchor=south east] at (20.5,-2) {
	\textsf{\textbf{ $n$-point correlation function}}
};

\node [boxBlue,R=15,C=105,align=center] at (10,-2) {\vspace{-4mm}
	\begin{align*}
	G^{(n)}_\mathrm{E}(x_1,\dots,x_n) 
	=  \left( \frac{1}{Z[J]} \frac{\delta}{\delta J(x_1)} \cdots \frac{\delta}{\delta J(x_n)} Z[J] \right)_{J=0}
	\end{align*}
};

\node [rotate=-90] at (10.25,-1.9) {\large $\implies$};

%%%%%%%%%%%%%%%%%%%%%%%%%%%%%%%%%%%%%%%%%%%%%%%%%%%%%%%%%%%%%%%%%%%%%%%%%%%%%%%
% C_F contour

\node (Cf) at (-9,4.75) {};
\def\px{Cf};

\draw[-triangle 45] ($(\px)-(2.5,0)$) -- ($(\px)+(3.5,0)$) 
	node[below] {\small $\op{Re} p^0$};
\draw[-triangle 45] ($(\px)-(0,2.5)$) -- ($(\px)+(0,2.5)$) 
	node[left,yshift=-2mm] {\small $\op{Im} p^0$};
\draw ($(\px)-(1.5,0)$) node[cross,red] {};
\draw ($(\px)-(1.5,0)$) node[below,red] {\small $-\omega_{\mathbf{p}}$};
\draw ($(\px)+(1.5,0)$) node[cross,blue] {};
\draw ($(\px)+(1.5,0)$) node[below,blue] {\small $+\omega_{\mathbf{p}}$};

% \draw[red, thick, dotted, decoration={
%         markings, mark=at position 0.8 with {\arrow[line width=1.5pt]{>}} 
%      }, postaction={decorate} ]
%   ($(\px)+(3,0)$) arc (0:-25:3 and -3);
% 
% \draw[blue, thick, dotted, decoration={
%         markings, mark=at position 0.8 with {\arrow[line width=1.5pt]{>}} 
%      }, postaction={decorate} ]
%   ($(\px)+(3,0)$) arc (0:25:3 and -3);

\draw[black!50!green, thick, decoration={
        markings, mark=at position 0.22 with {\arrow[line width=1.5pt]{>}},
        mark=at position 0.52 with {\arrow[line width=1.5pt]{>}},
        mark=at position 0.78 with {\arrow[line width=1.5pt]{>}} 
     }, postaction={decorate} ]
  ($(\px)-(2.5,0)$) 
  -- ($(\px)-(2.15,0)$) arc (0:180:-0.65 and -0.65)
  -- ($(\px)+(0.85,0)$) arc (0:180:-0.65 and 0.65)
  -- ($(\px)+(3.5,0)$);
  
\draw[black!50!green, thick, dashed, decoration={
        markings, mark=at position 0.8 with {\arrow[line width=1.5pt]{>}}
     }, postaction={decorate} ]
  ($(\px)+(-0.01,-2.5)$) -- ($(\px)+(-0.01,2.5)$);

\draw ($(\px)+(-0.45,1.2)$) node[black!50!green] {$C^\mathrm{E}_\mathrm{F}$};
\draw ($(\px)+(0.6,0.35)$) node[black!50!green] {$C_\mathrm{F}$};
% \draw ($(\px)+(2.2,1.6)$) node[red] {\small $x^0<0$};
% \draw ($(\px)+(2.2,-1.6)$) node[blue] {\small $x^0>0$};

\draw ($(\px)+(0.4,-1.5)$) node[black,text width=10,align=center] {\small 
	\begin{gather*}
		\text{rotation}\\[-0.4mm]
		\text{passive: }
			p^0 = \ii \, p^4, \\
		\text{active: }
			+\omega_\mathbf{p} = \ii \cdot (-\ii \omega_\mathbf{p})
	\end{gather*}
};

%%%%%%%%%%%%%%%%%%%%%%%%%%%%%%%%%%%%%%%%%%%%%%%%%%%%%%%%%%%%%%%%%%%%%%%%%%%%%%%
% C_F contour Euclidean

\node (Cfe) at (7.5,4.75) {};
\def\px{Cfe};

\draw[-triangle 45] ($(\px)-(2.5,0)$) -- ($(\px)+(3.5,0)$) 
	node[below] {\small $\op{Re} p^4$};
\draw[-triangle 45] ($(\px)-(0,2.5)$) -- ($(\px)+(0,2.5)$) 
	node[left,yshift=-2mm] {\small $\op{Im} p^4$};
\draw ($(\px)+(0,1.5)$) node[cross,red] {};
\draw ($(\px)+(0,1.5)$) node[left,red] {\small $+\ii\omega_{\mathbf{p}}$};
\draw ($(\px)+(0,-1.5)$) node[cross,blue] {};
\draw ($(\px)+(0,-1.5)$) node[left,blue] {\small $-\ii\omega_{\mathbf{p}}$};

% \draw[red, thick, dotted, decoration={
%         markings, mark=at position 0.8 with {\arrow[line width=1.5pt]{>}} 
%      }, postaction={decorate} ]
%   ($(\px)+(3,0)$) arc (0:-25:3 and -3);
 
% \draw[blue, thick, dotted, decoration={
%         markings, mark=at position 0.8 with {\arrow[line width=1.5pt]{>}} 
%      }, postaction={decorate} ]
%   ($(\px)+(3,0)$) arc (0:25:3 and -3);
 
\draw[black!50!green, thick, dashed, decoration={
        markings, mark=at position 0.7 with {\arrow[line width=1.5pt]{>}}
     }, postaction={decorate} ]
  ($(\px)+(-2.5,0.01)$) -- ($(\px)+(3.5,0.01)$);

\draw ($(\px)+(1.1,0.35)$) node[black!50!green] {$C^\mathrm{E}_\mathrm{F}$};
%\draw ($(\px)+(2.2,1.6)$) node[red] {\small $x^0<0$};
%\draw ($(\px)+(2.2,-1.6)$) node[blue] {\small $x^0>0$};

%%%%%%%%%%%%%%%%%%%%%%%%%%%%%%%%%%%%%%%%%%%%%%%%%%%%%%%%%%%%%%

\tikzset{myBox/.style={anchor=north west}}

\node [anchor=south west] at (-20.5,-4.5) {
	\textsf{\textbf{First-order expansion of the 2-point correlation function 
	$G^{(2)}(x_1,x_2)$ for the $\phi^4$-theory in the euclidean framework:}}
	(for E $\to$ L, replace $-\lambda \to \ii \lambda$)
};

\node [myBox,R=10,C=30,align=center] at (-20,-4.75) {\vspace{-3mm}
	\begin{gather*}\textstyle
	\frac{\delta}{\delta J(x_1)} \frac{\delta}{\delta J(x_2)} Z[J]
	\end{gather*}
};

\node [myBox,R=7.5,C=30,align=center] (s1) at (-20,-5.75) {\vspace{-5mm}
	\begin{gather*}\textstyle
	Z[J]
	\end{gather*}
};

\draw [thin] (s1.north west)--(s1.north east);

\node [myBox,R=10,C=70,align=center] at (-16.25,-4.75) {\vspace{-3mm}
	\begin{gather*}\textstyle
	\frac{\delta}{\delta J(x_1)} \frac{\delta}{\delta J(x_2)} 
	\exp\left( \,\boxed{- S_\mathrm{int}}\left[\frac{\delta}{\delta J(y)} \right] \right) Z_0[J]
	\end{gather*}
};

\node [myBox,R=10,C=70,align=center] (s2) at (-16.25,-5.75) {\vspace{-5mm}
	\begin{gather*}\textstyle
	\exp\left( \,\boxed{- S_\mathrm{int}}\left[\frac{\delta}{\delta J(y)} \right] \right) Z_0[J]
	\end{gather*}
};

\draw [thin] (s2.north west)--(s2.north east);

\node [myBox,R=10,C=145,align=center] at (-8.5,-4.75) {\vspace{-3mm}
	\begin{gather*}\textstyle
	\frac{\delta}{\delta J(x_1)} \frac{\delta}{\delta J(x_2)}
	\exp\left( - \frac{\lambda}{4!} \int\!\dd^4 y\, \frac{\delta^4}{\delta J(y)^4} \right)
	\left[ Z_0[0] \exp\left( \,\frac{1}{2}\int\!\dd^4 z_1 \, \dd^4 z_2\, J(z_1) \Delta^\mathrm{E}_\mathrm{F}(z_1\!-\!z_2) J(z_2) \right) \right]
	\end{gather*}
};

\node [myBox,R=10,C=145,align=center] (s3) at (-8.5,-5.75) {\vspace{-5mm}
	\begin{gather*}\textstyle
	\exp\left( - \frac{\lambda}{4!} \int\!\dd^4 y\, \frac{\delta^4}{\delta J(y)^4} \right)
	\left[ Z_0[0] \exp\left( \,\frac{1}{2}\int\!\dd^4 z_1 \, \dd^4 z_2\, J(z_1) \Delta^\mathrm{E}_\mathrm{F}(z_1\!-\!z_2) J(z_2) \right) \right]
	\end{gather*}
};

\draw [thin] (s3.north west)--(s3.north east);

\node at (-20.25,-5.78) {\large $=$};
\node at (-8.75,-5.78) {\large $=$};
\node at (-16.5,-5.75) {\large $=$};

\draw (-16.85,-5) -- (-16.85,-6.5) node [right] {\small $J=0$};
\draw (-9.1,-5) -- (-9.1,-6.5) node [right] {\small $J=0$};
\draw (6.15,-5) -- (6.15,-6.5) node [right] {\small $J=0$};

\node [anchor=north west,boxRed,R=22.5,C=130,align=justify] at (7.5,-4.5) {
	Now, Taylor expand the exponential function for the interaction to the required
	order in $\lambda$. Also, Taylor expand the exponential functions with
	the source to match the number of functional derivatives.
	Note that $J=0$ will cancel all unused source terms after the expansion
	(so they can be ignored).
};

\node [myBox,R=10,C=300,align=center] at (-20,-7) {\vspace{-3mm}
	\begin{gather*}\textstyle
	{\color{red}\frac{\delta}{\delta J(x_1)} \frac{\delta}{\delta J(x_2)}}
	\left( 1 - \frac{\lambda}{4!} \int\!\dd^4 y\, {\color{red}\frac{\delta^4}{\delta J(y)^4}} + \mathcal{O}(\lambda^2) \right)
	\left[ \text{(ign.\,$\textstyle \frac{1}{0!}$)}  
	+ \frac{1}{1!}\left(\frac{1}{2}\right)^1\!\int\!\dd^4 z_1 \, \dd^4 z_2\, {\color{red}J(z_1) J(z_2)} \, \Delta^\mathrm{E}_\mathrm{F}(z_1\!-\!z_2) 
	+ \text{(ign.\,$\textstyle \frac{1}{2!}$)}
	+ \frac{1}{3!}\left(\frac{1}{2}\right)^3\! \int\!\dd^4 z_1 \cdots \dd^4 z_6\, {\color{red}J(z_1)\cdots J(z_6) }
	\, \Delta^\mathrm{E}_\mathrm{F}(z_1\!-\!z_2) \Delta^\mathrm{E}_\mathrm{F}(z_3\!-\!z_4) \Delta^\mathrm{E}_\mathrm{F}(z_5\!-\!z_6) 
	\right]
	\end{gather*}
};

\node [myBox,R=10,C=300,align=center] (s4) at (-20,-8) {\vspace{-5mm}
	\begin{gather*}\textstyle
	\left( 1 - \frac{\lambda}{4!} \int\!\dd^4 y\, {\color{red}\frac{\delta^4}{\delta J(y)^4}}  +\mathcal{O}(\lambda^2) \right)
	\left[ 1
	+ \text{(ign.\,$\textstyle \frac{1}{1!}$)}
	+ \frac{1}{2!}\left(\frac{1}{2}\right)^2\!\int\!\dd^4 z_1 \cdots \dd^4 z_4\, \,{\color{red} J(z_1) \cdots J(z_4)} 
	\, \Delta^\mathrm{E}_\mathrm{F}(z_1\!-\!z_2)  \Delta^\mathrm{E}_\mathrm{F}(z_3\!-\!z_4)
	\right]
	\end{gather*}
};

\draw [thin] (s4.north west)--(s4.north east);

\node at (-20.25,-8.03) {\large $=$};

\node [anchor=north west,boxRed,R=20,C=82.5,align=left] at (12.25,-7.25) {
	Note:~ $ \tfrac{\delta^n}{\delta J(y)^n}
	J(z_1)\cdots J(z_n) = $\\[1mm]
	\hfill $= \sum_\sigma \delta(z_{\sigma(1)}-y) \cdots \delta(z_{\sigma(n)}-y)$,~\\[1mm]
	where $\sigma$ runs over $n!$ permuations.
};

\node [anchor=north west,boxRed,R=30,C=82.5,align=justify] at (12.25,-9.75) {
	Here, the disconnected loops in the denominator cancel out.
	The general result is the linked-cluster theorem which states
	that the vacuum diagrams cancel exactly out to all orders in perturbation theory.
	(The theoremis valid for all the $n$-point functions and for any local theory.)
};

\draw (10.15,-7.25) -- (10.15,-8.75) node [right] {\small $J=0$};

\node [myBox,R=10,C=265,align=center] at (-20,-9) {\vspace{-3mm}
	\begin{gather*}\textstyle
	\frac{1}{1!}\left(\frac{1}{2}\right)^1 \!\!\times 2 \!\times\! \Delta^\mathrm{E}_\mathrm{F}(x_1\!-\!x_2)
	+ \frac{1}{3!}\left(\frac{1}{2}\right)^3 \!\!\times\! 6 \!\times\! 4! \!\times\!
	\left( - \frac{\lambda}{4!} \int\!\dd^4 y \right)
	\Delta^\mathrm{E}_\mathrm{F}(x_1\!-\!x_2) \Delta^\mathrm{E}_\mathrm{F}(y\!-\!y) \Delta^\mathrm{E}_\mathrm{F}(y\!-\!y)
	+ \frac{1}{3!}\left(\frac{1}{2}\right)^3 \!\!\times\! 6 \!\times\! 4 \!\times\! 4! \!\times\!
	\left( - \frac{\lambda}{4!} \int\!\dd^4 y \right)
	\Delta^\mathrm{E}_\mathrm{F}(x_1\!-\!y) \Delta^\mathrm{E}_\mathrm{F}(y\!-\!x_2) \Delta^\mathrm{E}_\mathrm{F}(y\!-\!y)
	 + \mathcal{O}(\lambda^2)
	\end{gather*}
};

\node [myBox,R=10,C=265,align=center] (s5) at (-20,-10) {\vspace{-6mm}
	\begin{gather*}\textstyle
	1 + \frac{1}{2!}\left(\frac{1}{2}\right)^2 \!\!\times\! 4! \!\times\!
	\left( - \frac{\lambda}{4!} \int\!\dd^4 y \right)
	\Delta^\mathrm{E}_\mathrm{F}(y\!-\!y) \Delta^\mathrm{E}_\mathrm{F}(y\!-\!y)
	+\mathcal{O}(\lambda^2)
	\end{gather*}
};

\draw [thin] (s5.north west)--(s5.north east);

\node at (-20.25,-10.03) {\large $=$};

\node [myBox,R=10,C=205,align=center] at (-20,-11) {\vspace{-3mm}
	\begin{gather*}\textstyle
	\Delta^\mathrm{E}_\mathrm{F}(x_1\!-\!x_2)
	+ 3 \!\times\! \Delta^\mathrm{E}_\mathrm{F}(x_1\!-\!x_2) 
	\left( - \frac{\lambda}{4!} \int\!\dd^4 y \right)
	\Delta^\mathrm{E}_\mathrm{F}(y\!-\!y) \Delta^\mathrm{E}_\mathrm{F}(y\!-\!y)
	+ 12 \!\times\!
	\left( - \frac{\lambda}{4!} \int\!\dd^4 y \right)
	\Delta^\mathrm{E}_\mathrm{F}(x_1\!-\!y) \Delta^\mathrm{E}_\mathrm{F}(y\!-\!x_2) \Delta^\mathrm{E}_\mathrm{F}(y\!-\!y)
	 + \mathcal{O}(\lambda^2)
	\end{gather*}
};

\node [myBox,R=10,C=205,align=center] (s6) at (-20,-12) {\vspace{-6mm}
	\begin{gather*}\textstyle
	1 + 3 \!\times\!
	\left( - \frac{\lambda}{4!} \int\!\dd^4 y \right)
	\Delta^\mathrm{E}_\mathrm{F}(y\!-\!y) \Delta^\mathrm{E}_\mathrm{F}(y\!-\!y)
	+\mathcal{O}(\lambda^2)
	\end{gather*}
};

\draw [thin] (s6.north west)--(s6.north east);

\node at (-20.25,-12.03) {\large $=$};

\node [myBox,R=12.5,C=114,align=center] at (0.5,-11.35) {\vspace{-4mm}
	\begin{gather*}
	= \Delta^\mathrm{E}_\mathrm{F}(x_1\!-\!x_2)
	- \frac{\lambda}{2} \!\int\!\dd^4 y \,
	\Delta^\mathrm{E}_\mathrm{F}(x_1\!-\!y) \Delta^\mathrm{E}_\mathrm{F}(y\!-\!x_2) \Delta^\mathrm{E}_\mathrm{F}(y\!-\!y)
	 + \mathcal{O}(\lambda^2) \,,
	\end{gather*}
};

\node [myBox,R=15,C=30,align=left] at (-20.5,-13) {
	\vspace{-3.5mm}
	\begin{align*} \implies
	G^\mathrm{E}(p) = G^\mathrm{E}_0(p) - \frac{\lambda}{2} \, G^\mathrm{E}_0(p)^2 \Delta^\mathrm{E}_\mathrm{F}(0)  + \mathcal{O}(\lambda^2) 
	= \frac{1}{p^2+m^2} \left( 1 - \frac{\lambda}{2} \frac{1}{p^2+m^2} \int \frac{\dd^4 q}{(2\pi)^4} \frac{1}{q^2+m^2} \right)  + \mathcal{O}(\lambda^2)
	= \frac{1}{p^2+m^2 + \frac{\lambda}{2} \int \frac{\dd^4 q}{(2\pi)^4} \frac{1}{q^2+m^2}} + \mathcal{O}(\lambda^2),
	\qquad \text{with hard cut-off:~~}
	\delta m^2 = \frac{\lambda}{2} \int^\Lambda_0 \frac{\dd^4 q}{(2\pi)^4} \frac{1}{q^2+m^2}
	= \frac{\lambda}{32\pi^2} \left( \Lambda^2 - m^2 \log \frac{\Lambda^2+m^2}{m^2} \right) \!.
	\end{align*}
};

%%%%%%%%%%%%%%%%%%%%%%%%%%%%%%%%%%%%%%%%%%%%%%%%%%%%%%%%%%%%%%

\end{tikzpicture}

%%%%%%%%%%%%%%%%%%%%%%%%%%%%%%%%%%%%%%%%%%%%%%%%%%%%%%%%%%%%%%

\end{document}